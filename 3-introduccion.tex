\chapter{Introducción}
\label{introduccion}

% On April 27, 1900 William Thomson (Lord Kelvin) gave his “Two Clouds” speech (“Nineteenth‐Century
% Clouds over the Dynamical Theory of Heat and Light”) at the Royal Institution, in which he argued that
% “The beauty and clearness of the dynamical theory, which asserts heat and light to be modes of motion, is
% at present obscured by two clouds.” The two open problems in physics that Kelvin referred to were the
% failure of the Michelson‐Morley experiment to detect the luminous ether (“how could the earth move
% through an elastic solid, such as essentially is the luminiferous ether?”), and the ultraviolet paradox
% (“the Maxwell‐Boltzmann doctrine regarding the partition of energy”). Within a decade, Einstein had
% proposed fundamentally novel insights that led to two paradigm shifts that define modern physics to this
% day—the transformation of these two “clouds” into relativity and quantum mechanics.
% In 1987, Keith Beven gave what might be considered hydrology's version of the Two Clouds speech at a
% symposium of the International Association of Hydrological Sciences (IAHS) (Beven, 1987). He took a
% perspective inspired by Thomas Kuhn's theory of scientific revolutions (Kuhn, 1962) to argue that “[t]he
% extension of laboratory scale theory to the catchment scale is unjustified and that a radical change in
% theoretical structure (a new paradigm) will be required before any major advance can be made in [predicting
% catchment‐scale rainfall‐runoff responses].” He proposed that two things would be necessary to push the
% field of surface hydrology into a new period of “normal science”: (i) scale‐relevant theories of watersheds
% (“[h]ydrology in the future will require a macroscale theory that deals explicitly with the problems posed
% by spatial integration of heterogeneous nonlinear interacting processes”) and (ii) uncertainty quantification
% (“[s]uch a theory will be inherently stochastic and will deal with the value of observations and qualitative
% knowledge in reducing predictive uncertainty.”)
% Unfortunately, hydrology has not had its Einstein (with all due respect to Einstein, 1926, 1950). Nine decades
% from the establishment of the Hydrology section of the American Geophysical Union and after more than a
% half‐century of computer‐based hydrological modeling (Crawford & Burges, 2004), Blöschl et al. (2019) listed
% as one of the 23 “Unsolved Problems in Hydrology”: “what are the hydrologic laws at the catchment scale
% and how do they change with scale?”

% --El problema de la hidrología a gran escalas
\section{Estado del arte}

Un problema de larga data en la hidrología a gran escala es el hecho de que los modelos tradicionales, ya sean agregados o basados 
en procesos físicos, en general no son extrapolables en el espacio, es decir, un modelo calibrado en una cuenca dada no es 
válido para otra. El desafío principal que deben sortear estos modelos es el de aprender a codificar las diferencias 
en las características locales para poder extrapolar información hidrológica 
de un región a otra, por ejemplo de cuencas calibradas a cuencas sin calibrar 
\cite{Bloschl2} \cite{Razavi} \cite{Hrachowitz} \cite{Bloschl3}.   

Una pregunta a responder es si realmente existe una teoría hidrológica válida a macro escalas que englobe la cantidad de 
procesos heterogéneos que ocurren en una cuenca hidrográfica \cite{nearing}. Más aun, en el artículo \textit{Twenty-three Unsolved Problems 
in Hydrology (UPH)- a community perspective} 
publicado en el año 2019 \cite{Bloschl} se ha incluido la pregunta de 
``\textit{¿Cuáles son las leyes hidrológicas válidas en las  escalas de una cuenca hidrográfica?}''
entre uno de los 23 problemas sin resolver en la hidrología. La respuesta a esta pregunta probablemente venga de la mano
los avances de las técnicas de aprendizaje automático y aprendizaje profundo (ML y DP por sus siglas en inglés) que 
indican que es posible derivar un modelo global a partir de los datos
observacionales disponibles que funcione en cuencas hidrológicas a gran escala\cite{nearing}. 


Éstos modelos derivados de los datos, que carecen de un modelo físico conceptual asociado, no se basan en el conocimiento 
previo del sistema hidrológico, sino que aprenden los diferentes comportamientos de las cuencas directamente de 
las entradas meteorológicas y funcionan mejor cuando se entrenan en una gran cantidad de datos proveniente de diferentes cuencas 
\cite{Gupta}\cite{Peters}. 
Por el contrario, los modelos tradicionales funcionan mejor cuando se calibran individualmente en cada cuenca y los resultados 
obtenidos no son transferibles a otras regiones.  

Estudios recientes han demostrado que los modelos
de DL entrenados en cuencas calibradas predicen mejor en cuencas no calibradas que los modelos tradicionales.
Kratzert et al. \cite{Kratzert2} demostraron que las redes neuronales que incluyen celdas con memoria a largo plazo 
(LSTM por sus siglas en inglés) superan la performance del bien conocido modelo de sacramento (SAC-SMA) \cite{sac1} \cite{sac2}
en una serie de cuencas para las cuales el modelo SAC-SMA había sido calibrado de manera individual.  
El modelo LSTM fue capaz de realizar mejores predicciones del caudal en dichas cuencas siendo que este nunca antes había visto
los datos de entrenamiento de estas cuencas. 

Un estudio más reciente\cite{nearing}\cite{Kratzert}, en donde se han utilizado datos provenientes de varios cientos de cuencas 
continentales en los Estados Unidos a lo largo de 30 años, demuestra que los modelos de aprendizaje profundo arrojan 
mejores predicciones para los caudales de descarga en cuencas no aforadas que los obtenidos por modelos tradicionales 
calibrados con registros de datos en cuencas aforadas. 
Más aun, estos estudios han demostrado que las redes LSTM son incluso capaces de modelar la presencia de nieve y 
almacenar esta información en las celdas específicas de memoria sin siquiera haber sido entrenadas directamente en 
ningún tipo de observación relacionada con nieve más allá de la precipitación total y la temperatura \cite{Kratzert2}.
A partir de estos resultados, se puede presuponer que este enfoque de desarrollar
un modelado global es prometedor. 




%En un estudio preliminar Kratzert et al. (2018c) demostraron que su enfoque
% %           basado en LSTM supera, en promedio, el Modelo de Contabilidad de la Humedad del Suelo de Sacramento 
% %           (SAC-SMA) bien calibrado en una comparación asimétrica en la que se usó el LSTM en un entorno sin medir
% %            y SAC-SMA se usó en un configuración calibrada, es decir, SAC-SMA se calibró individualmente para 
% %            cada cuenca, mientras que el LSTM nunca vio datos de entrenamiento de ninguna cuenca donde se usó 
% %            para la predicción. Esto se hizo proporcionando al modelo basado en LSTM datos de forzamiento
% %             meteorológico y atributos de cuenca adicionales. A partir de estos resultados preliminares ya 
% %             podemos suponer que este enfoque de modelado general es prometedor y tiene potencial para la regionalización.

% --apricion de ML DL
% Con el avance de la mchine learning y el deep learning indican que se podria dirivar
% u modelo o teria pra las cuencas a macro escala de los datos observacionales disponibles.
% Se puede aprende un modelo general con DL. Recientes estudios han demostrado que los modelos de DL entrenados en cuencas calibradas
% predicen mejor en cuencas no calibradas que modelos tradicionales.
% DL funciona mejor cuando se entrena en muchas cuencas, pero los modelos tradicionales funcionan mejor cuando se calibran individualmente.

% Los métodos independientes del modelo, por el contrario, no se basan en el conocimiento previo del sistema hidrológico. 
% En cambio, estos métodos aprenden todo el mapeo desde datos auxiliares y entradas meteorológicas hasta flujos de corriente
%  u otros flujos de salida directamente. Un modelo de este tipo tiene que "aprender" cómo los atributos de la cuenca u otros
%   datos auxiliares distinguen entre diferentes comportamientos de respuesta de la cuenca. Sin embargo, el modelado hidrológico
%    generalmente proporciona las predicciones más precisas cuando un modelo se calibra para una sola cuenca específica
%     (Mizukami et al., 2017), mientras que los enfoques basados ​​en datos pueden

%beneficiarse de una gran sección 
%     transversal de diversos datos de entrenamiento, porque el conocimiento puede ser transferido a través de los 
%     sitios. Entre la categoría de enfoques basados ​​en datos se encuentran las redes neuronales. Besaw et al. (2010)
%      demostraron que una red neuronal artificial entrenada en una cuenca (usando solo datos meteorológicos) podría
%       trasladarse a una cuenca similar (durante un período de tiempo similar). Sin embargo, la precisión de su red
%     %    en la cuenca de entrenamiento fue solo un NSE de 0,29. Recientemente, Kratzert et al. (2018b) han demostrado 
% %        que las redes de memoria a corto plazo (LSTM), un tipo especial de red neuronal recurrente, son adecuadas
% %         para la tarea de modelado de lluvia y escorrentía. Este estudio ya incluyó los primeros experimentos
% %          hacia el modelado regional mientras todavía usaba solo datos meteorológicos e ignoraba los atributos 
% %          de cuencas auxiliares. En un estudio preliminar Kratzert et al. (2018c) demostraron que su enfoque
% %           basado en LSTM supera, en promedio, el Modelo de Contabilidad de la Humedad del Suelo de Sacramento 
% %           (SAC-SMA) bien calibrado en una comparación asimétrica en la que se usó el LSTM en un entorno sin medir
% %            y SAC-SMA se usó en un configuración calibrada, es decir, SAC-SMA se calibró individualmente para 
% %            cada cuenca, mientras que el LSTM nunca vio datos de entrenamiento de ninguna cuenca donde se usó 
% %            para la predicción. Esto se hizo proporcionando al modelo basado en LSTM datos de forzamiento
% %             meteorológico y atributos de cuenca adicionales. A partir de estos resultados preliminares ya 
% %             podemos suponer que este enfoque de modelado general es prometedor y tiene potencial para la regionalización.



% ​en datos.



% En el año 2019 salió publicado un artículo titulado \textit{Twenty-three Unsolved Problems in Hydrology (UPH)- a community perspective} \cite{Bloschl}
% en donde la pregunta "\textit{¿Cuáles son las leyes hidrológicas en las macro escalas de una cuenca?}"
% aparece entre los 23 problemas sin resolver en la hidrología. Hay varias potenciales razones por las cuales aún hoy en día
% no existe una teoría a macro escalas relevante en hidrología, una de ellas es que tal teoria no existe y que a tales 
% escalas el comportamiento de las cuencas se encuentra dominado por una cantidad heterogenea de procesos que se traducen 
% poca consistencia a través de las diferentes cuencas. O tambien prodia ser que sí existen patrones a macro escala pero que carecemos 
% de informacion suficiente para encontrarlos. 

% ,
% ya que hoy en día los modelos hidrológicos más exitosos se calibran en una cuenca específica mientras que un modelo 
% debe ser capaz de describir los procesos que ocurren en grandes extensiones abarcando diferentes sub-cuencas. 
% El desafío en diferentes modelos es aprender a codificar estas diferencias de manera de poder abarcar las diferencias en las 
% características locales y describir un comportamiento hidrológico heterogéneo. 


% Un problema de larga data en la hidrologia es como usar un modelo o conjunto de modelos para proveer una simulacion hidrologica
% espacialmente continua atra ves de grandes areas. Este es el problema conocido como el problema del modelado
% regional. El desafio principal es como extrapolar informacion hidrologica de un area a otra, por ejemplo 
% cuancas calibradas a cuencas sin calibrar (Blöschl and Sivapalan, 1995). Este problema se encuentra cercananmente relacionado
% con el problema de prediccion en cuencas sin calibrar.  Razavi and Coulibaly (2013)  Hrachowitz et al. (2013) Prieto et al. (2019).

% Hoy en día los modelos hidrologicos mas exitosos se encuentran calibrados a una cuenca espesifica 
% mientras que que un modelo debe tener en cuenta los diferentes comportamientos en diferentes cuencas.

% el desafio en diferentes modelos es aprender a codificar estas diferencias de manera de las diferencias en las caracteristicas de las 
% subcuencas se puedan trasladar en el comportamiento hidrologico eterogeneo.

% hay dos estrategias principales para modelado regional, los metodos que dependen del modelo 
% y los metodos que se derivan de los datos que no incluyen un modelo especifico.

% \cite{chambo}
% \section{Motivación}
% Entrenar un modelo LSTM en la cuenca hidrografica de chambo y ver si es capaz de explicar las caracteristicas locales de cada subcuenca.


\section{Objetivos y metodología}

En este trabajo se han desarrollado modelos de redes neuronales (ANNs) capaces de aprender las diferentes características presentes a lo largo
de las 76 sub-cuencas que conforman la cuenca del río Chambo (CHRC) directamente a partir de datos meteorológicos y predecir los caudales de 
descarga en cada una de ellas. 
El objetivo del trabajo es evaluar el desempeño de estos modelos en un entorno calibrado, es decir que no se le pedirá
a los modelos que predigan en cuencas para las cuales nunca han visto sus datos de entrenamiento, 
y  determinar si estos modelos, que utilizan datos  meteorológicos, 
son capaces de combinar diferentes partes de la red para predecir los caudales en diferentes puntos a lo largo de toda la cuenca. 

En la primera parte del trabajo, el propósito fue generar simulaciones hidrológicas para los caudales naturales de las cuencas 
que fueron utilizados como las variables objetivo para entrenar los modelos de ANNs. Con este fin se ha utilizado el modelo 
desarrollado por IHCantabria, denominado MELCA (Modelo de Equilibrio Logístico para Cuencas Andinas) descrito en el capítulo \ref{Modelo_Hidrológico} 
que constituye una adaptación del modelo general denominado LEM (Modelo de Equilibrio Logístico) para cuencas 
de alta montaña con presencia de páramos y bofedales. Las series meteorológicas utilizadas para simular los caudales con el modelo hidrológico 
y entrenar los modelos de ANNs son descritas en el capítulo \ref{datos}. Estas abarcan un período temporal de 20 años e incluyen series de 
precipitación, series de temperatura máxima y temperatura mínima.  

Una vez generados los valores de los caudales simulados por el modelo hidrológico se procedió a entrenar tres configuraciones
diferentes de modelos de ANNs descritos en el capítulo \ref{ANNs}. Para esto, se han divido los datos en conjuntos de entrenamiento y 
validación y se han re-estructurado para que puedan ser utilizados como inputs en las diferentes estructuras de redes consideradas. 
Para validar los modelos se ha utilizado demás de la función de loss, incluida en los algoritmos de las ANNs, el coeficiente de
Nash-Stutcliffe (NSE) que es ampliamente utilizado en hidrología. Con el fin de determinar 
la influencia que los restos o diferencias entre los caudales predichos por las ANNs y los caudales simulados con 
el modelo MELCA ejercen al determinar el balance hídrico en CHRC, se ha utilizado el software MODSIM que permite 
optimizar la distribución del agua sobre los cauces de la cuenca teniendo en cuenta los diferentes usos del agua.

Por último, y con el objetivo de poder ejecutar los modelos desarrollados de manera sencilla en otras cuencas hidrográficas, se ha creado
una aplicación REST APIs  utilizando la librería flask de python. Los detalles del proceso de desarrollo se describen en el anexo \ref{Desarrollo}.










% --que se hizo en este trabajo


% El problema de prediccion en cuencas no calibradas, es funcamentalmente un problema de extrapolacion. 

% -- tipos de modelos clasicos

% Modelos agregados, conceptuales o de caja gris. También son
% modelos reducibles a series temporales (es decir, admiten una
% formulación state-space), tienen una base conceptual mecanicista.
% • Modelos basados en procesos físicos. Son los modelos que aspiran
% a resolver en un dominio de cálculo 3D las ecuaciones que rigen los
% procesos físicos fundamentales del suelo, su superficie y la capa de
% aire superior (Navier-Stokes + leyes de la termodinámica).


% los modelos agregados Se basan en un marco conceptual mecanicista. Suelen hacer uso de una
% estructura de depósitos conceptuales, en serie y en paralelo, como
% representación abstracta y simplificada de la cuenca.
% • Son metáforas matemáticas del territorio con pocos grados de libertad
% (típicamente 4-7 parámetros). Sus parámetros pueden tener un sentido
% físico, pero no son parámetros físicos: no pueden medirse in situ y deben
% de obtenerse mediante calibración a partir de series observadas.
% • Al tener pocos parámetros (parsimonia), se pueden calibrar con relativa
% facilidad, empleando métodos automáticos o semi-automáticos.
% • Suelen ser de código abierto y son computacionalmente eficientes, lo
% que permite explotarlos en continuo (no eventos aislados) con series
% largas, en sistemas en tiempo real y/o de forma probabilista.
% • De forma general, no son extrapolables en el espacio (el modelo de una
% cuenca no es válido para otra). En las cuencas donde no existen aforos,
% se requieren técnicas de regionalización.





% \section{Resultados obtenidos}

% Resultados obtenidos

% \section{Organización del documento}

% Organización del documento

% \begin{itemize}
%     \item Capítulo 1.
%     \item Capítulo 2.
%     \item Capítulo 3.
% \end{itemize}