\chapter{Introducción}
\label{introduccion}

% On April 27, 1900 William Thomson (Lord Kelvin) gave his “Two Clouds” speech (“Nineteenth‐Century
% Clouds over the Dynamical Theory of Heat and Light”) at the Royal Institution, in which he argued that
% “The beauty and clearness of the dynamical theory, which asserts heat and light to be modes of motion, is
% at present obscured by two clouds.” The two open problems in physics that Kelvin referred to were the
% failure of the Michelson‐Morley experiment to detect the luminous ether (“how could the earth move
% through an elastic solid, such as essentially is the luminiferous ether?”), and the ultraviolet paradox
% (“the Maxwell‐Boltzmann doctrine regarding the partition of energy”). Within a decade, Einstein had
% proposed fundamentally novel insights that led to two paradigm shifts that define modern physics to this
% day—the transformation of these two “clouds” into relativity and quantum mechanics.
% In 1987, Keith Beven gave what might be considered hydrology's version of the Two Clouds speech at a
% symposium of the International Association of Hydrological Sciences (IAHS) (Beven, 1987). He took a
% perspective inspired by Thomas Kuhn's theory of scientific revolutions (Kuhn, 1962) to argue that “[t]he
% extension of laboratory scale theory to the catchment scale is unjustified and that a radical change in
% theoretical structure (a new paradigm) will be required before any major advance can be made in [predicting
% catchment‐scale rainfall‐runoff responses].” He proposed that two things would be necessary to push the
% field of surface hydrology into a new period of “normal science”: (i) scale‐relevant theories of watersheds
% (“[h]ydrology in the future will require a macroscale theory that deals explicitly with the problems posed
% by spatial integration of heterogeneous nonlinear interacting processes”) and (ii) uncertainty quantification
% (“[s]uch a theory will be inherently stochastic and will deal with the value of observations and qualitative
% knowledge in reducing predictive uncertainty.”)
% Unfortunately, hydrology has not had its Einstein (with all due respect to Einstein, 1926, 1950). Nine decades
% from the establishment of the Hydrology section of the American Geophysical Union and after more than a
% half‐century of computer‐based hydrological modeling (Crawford & Burges, 2004), Blöschl et al. (2019) listed
% as one of the 23 “Unsolved Problems in Hydrology”: “what are the hydrologic laws at the catchment scale
% and how do they change with scale?”

\cite{chambo}
\section{Motivación}

Motivación de la tesis.

Un ejemplo de cita.\cite{web:unlp}

\section{Objetivos y metodología}

Objetivos y metodología.

\section{Resultados obtenidos}

Resultados obtenidos

\section{Organización del documento}

Organización del documento

\begin{itemize}
    \item Capítulo 1.
    \item Capítulo 2.
    \item Capítulo 3.
\end{itemize}