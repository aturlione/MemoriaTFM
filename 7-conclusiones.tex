\chapter{Discusiones y Conclusiones}
\label{Discusiones y conclusiones}

Los resultados descritos en la sección anterior demuestran que el modelo que realiza los mejores ajustes es el LSTM1 cuando es entrenado en 
el espacio de las componentes principales. Aproximadamente un 70$\%$ de los ajustes obtenidos con el mismo son excelentes mientras que esta cifra  
desciende al 60$\%$ para los modelos denso y LSTM1 entrenado localmente. Si bien los últimos modelos poseen una buena performance en la mayoría de 
las sub-cuencas, presentan mucha variación en el coeficiente de NSE, alcanzando incluso valores negativos en algunos puntos para los cuales 
el modelo LSTM1 PCA arroja ajustes que son desde aceptables hasta excelentes. Por otro lado el experimento realizado con el modelo LSTM2, que considera 
un entrenamiento secuencial utilizando solamente los valores de la serie de caudales de descarga de las sub-cuencas, ha demostrado
tener una  mala performance en la mayoría de los puntos. 

Los resultados antes mencionados nos llevan a concluir que la inclusión de las celdas de memoria LSTM y la presencia de componentes principales 
mejoran notablemente la performance de los modelos en cuencas hidrográficas. 
Por un lado las celdas LSTM permiten almacenar información de eje temporal sobre los diferentes procesos  que ocurren en  las cuencas,
y así captar mas información sobre la relación entre eventos de precipitación y descarga. Por otro lado, 
las componentes principales reducen considerablemente la dimensión del espacio predictor
y combinan la información proveniente de todo el dominio del mismo, lo que soluciona problemas de sobre ajustes y el problema
des desvanecimiento del loss presente en las cuencas más pequeñas y áridas. 
Por último, se ha demostrado que los resultados obtenidos con redes neuronales
pueden ser utilizados para simular operaciones en la cuenca Chambo, ya que se ha comprobado que
el resultado del balance hidrológico obtenido con Modsim son los mismo que los obtenidos con los valores simulados por el modelo LEM.


Estos resultados concuerdan con estudios anteriores en los cuales se concluye que 
las redes neuronales generalmente requieren una gran cantidad de datos de entrenamiento y que 
los ajustes que se obtienen al entrenar modelos de aprendizaje profundo en una sola sub-cuenca no suelen ser fiables \cite{Kratzert}. 
Esto supone una gran diferencia con el modelado y calibrado hidrológico tradicional que normalmente obtiene 
los mejores resultados cuando los modelos se calibran de forma independiente para cada sub-cuenca.
Esta propiedad de los modelos clásicos presenta problemas, ya que se ha observado que los 
 parámetros obtenidos por extrapolaciones  basadas en valores calibrados en cuencas de referencia 
pueden dar lugar a espacios de parámetros poco realistas\cite{Mizukami}. 
Los modelos LSTM en cambio, demuestran tener la capacidad de aprender simultáneamente relaciones de series temporales 
y espaciales en el mismo marco predictivo, lo que evita muchos problemas que actualmente se encuentran 
asociados con la estimación y transferencia de parámetros de modelos hidrológicos tradicionales \cite{Kratzert}, \cite{nearing}.


Un conclusión importante es entonces que las celdas LSTM son capaces de generar un modelo único a partir de grandes conjuntos de datos de 
que sea capaz de reflejar los comportamientos hidrológicos regionales específicos de cada sub-cuenca
ya que estos modelos son capaces de vincular las características locales de las sub-cuencas y aprender 
un modelo general a partir de los datos combinados de todas ellas. 
Más aún, existe un estudio donde se han utilizado datos provenientes de varios cientos de cuencas continentales en los Estados Unidos
a lo largo de 30 años, que demuestra que los modelos de aprendizaje profundo arrojan mejores predicciones para los caudales de descarga  
en cuencas no aforadas que los obtenidos por modelos tradicionales calibrados con registros de datos en cuencas aforadas\cite{nearing}\cite{Kratzert}.
Es por esto que concluimos que la principal virtud de las redes neuronales no es simplemente el hecho de que ajusten bien sino su capacidad de
aprendizaje y  flexibilidad para ser utilizadas en una variedad de  lugares y condiciones diferentes.

