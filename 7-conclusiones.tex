\chapter{Discusiones y Conclusiones}
\label{Discusiones y conclusiones}

Los resultados descritos en el capítulo anterior demuestran que el modelo que realiza los mejores ajustes es el LSTM1 cuando es entrenado en 
el espacio de las componentes principales. Aproximadamente un 70$\%$ de los ajustes obtenidos con el mismo son excelentes mientras que esta cifra  
desciende al 60$\%$ para los modelos denso y LSTM1 entrenado localmente. Si bien los últimos modelos poseen una buena performance en la mayoría de 
las sub-cuencas, presentan mucha variación en el coeficiente de NSE, alcanzando incluso valores negativos en algunos puntos para los cuales 
el modelo LSTM1 PCA arroja resultados que son desde aceptables hasta excelentes. Por otro lado el experimento realizado con el modelo LSTM2, que considera 
un entrenamiento secuencial utilizando solamente los valores de la serie de caudales de descarga de las sub-cuencas, ha demostrado
tener una  mala performance en la mayoría de los puntos. 

Los resultados antes mencionados nos llevan a concluir que la inclusión de las celdas de memoria LSTM y la presencia de componentes principales 
mejoran notablemente la performance de los modelos en cuencas hidrográficas. 
Por un lado las celdas LSTM permiten almacenar información de eje temporal sobre los diferentes procesos  que ocurren en  las sub-cuencas,
y así captar más información sobre la relación entre eventos de precipitación y descarga. Por otro lado, 
las componentes principales reducen considerablemente la dimensión del espacio predictor
y combinan la información proveniente de todo el dominio del mismo, lo que soluciona problemas de sobre ajustes y el problema
des desvanecimiento del loss presente en las cuencas más pequeñas y áridas. 
Finalmente, una vez entrenados lo modelos, se ha demostrado que los resultados obtenidos con redes LSTM
pueden ser utilizados para simular operaciones en la cuenca CHRC, ya que se ha comprobado que
el resultado del balance hidrológico obtenido con MODSIM es el mismo que los obtenidos con los valores simulados por el modelo LEM.

Todos estos resultados apuntan en la misma línea que estudios recientes en los cuales se concluye que 
las redes neuronales generalmente requieren una gran cantidad de datos de entrenamiento y que 
los ajustes que se obtienen al entrenar modelos de aprendizaje profundo en una sola sub-cuenca no suelen ser fiables \cite{Kratzert}. 
Esto supone una gran diferencia con el modelado y calibrado hidrológico tradicional que normalmente demuestra una mejor performance 
cuando los modelos se calibran de forma independiente para cada sub-cuenca.
Ésta propiedad de los modelos clásicos presenta problemas, ya que se ha observado que los 
 parámetros obtenidos por extrapolaciones  basadas en valores calibrados en cuencas de referencia 
pueden dar lugar a espacios de parámetros poco realistas\cite{Mizukami}. 
Los modelos LSTM en cambio, demuestran tener la capacidad de aprender simultáneamente relaciones de series temporales 
y espaciales en el mismo marco predictivo, lo que evita muchos problemas que actualmente se encuentran 
asociados con la estimación y transferencia de parámetros de modelos hidrológicos tradicionales \cite{Kratzert}, \cite{nearing}.

Una conclusión importante es entonces que las celdas LSTM son capaces de generar un modelo único a partir de grandes conjuntos de datos  
capaz de reflejar los comportamientos hidrológicos regionales específicos de cada sub-cuenca
ya que estos modelos  vinculan las características locales de las sub-cuencas y aprenden
un modelo general a partir de los datos combinados de todas ellas. 
Es por esto que concluimos que la principal virtud de las redes neuronales no es simplemente el hecho de que ajusten bien sino su capacidad de
aprendizaje y  flexibilidad para ser utilizadas en una variedad de  lugares y condiciones diferentes.


Por último, como posibles mejoras para trabajos futuros se propone concatenar los descriptores estáticos de las sub-cuencas, 
como por ejemplo el área, la longitud del cauce, etc., al espacio predictor. Otra mejora podría ser definir una función objetivo al 
entrenar los modelos que no dependa del valor medio de los caudales de descarga de las diferentes cuencas. Kratzert etal. \cite{Kratzert} 
proponen por ejemplo usar directamente una definición global del coeficiente de Nash-Stutcliffe durante el entrenamiento
de los datos. En este caso, si bien se perdería la linelidad entre las métricas del RMSE y el NSE,  esto permitiría 
contemplar el hecho de que las variancias de los datos observacionales difieren en distintas cuencas y así evitar el sobre-peso 
asociado a las cuencas más grandes y húmedas.
