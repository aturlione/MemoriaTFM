\setcounter{page}{1}
\chapter*{Resumen}
\addcontentsline{toc}{chapter}{Resumen}

En este trabajo se ha evaluado el desempeño de diferentes modelos de redes neuronales
para predecir el caudal natural en la cuenca hidrográfica Chambo  a partir de series de
entrada hidro-climáticas que abarcan un rango de 14 años. Se han considerado redes con
diferentes estructuras y se han entrenado de manera local, global y secuencial. Se ha medido el
desempeño de los modelos utilizando diferentes métricas y se ha comprobado que en la
mayoría de los casos los modelos logran reconstruir de manera correcta los caudales en un
rango de 6 años. Con los resultados obtenidos se ha utilizado el software de gestión MODSIM
que permite optimizar los usos de los recursos de la cuenca y se ha comprobado que el balance hídrico 
obtenido con los caudales predichos por los modelos de redes neuronales es el mismo que el obtenidos con un
modelo hidrológico tradicional. Finalmente se ha desarrollado una aplicación que permite la ejecución del modelo
en diferente cuencas hidrográficas.\\
\\
\textbf{Palabras claves:} Redes Neuronales, redes recurrentes, hidrología, predicción, cuencas hidrográficas, aprendizaje automático, aprendizaje profundo\\
\\
\\
\large{\textbf{Abstract}}
\vspace{5mm}

In this work, the performance of different neural network models
to predict the natural flow in the  watershed Chambo  has been evaluated. 
Models with different configurations  have been trained locally, globally and sequentially  in 
hydro-climatic series spanning a 14-year range.
The performance of the models, has been measured
using different metrics, and it has been verified that in most cases, they succeed in correctly reconstructing the flows over a
6 year range. The obtained results  has been used to optimize the use of the resources of the basin with the software MODSIM, 
and it has been proven that the results obtained by means of neural network models are the same as those obtained with a
traditional hydrological model. Finally, an application has been developed that allows the execution of the model
in different hydrological basins.\\
\\
\textbf{Key words:} Neural Networks, recurrent networks, hydrology, prediction, hydrographic basin, machine learning, deep learning