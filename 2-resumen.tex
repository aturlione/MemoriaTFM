\setcounter{page}{1}
\chapter*{Resumen}
\addcontentsline{toc}{chapter}{Resumen}

En este trabajo se ha evaluado el desempeño de diferentes modelos de redes neuronales (RN)
para predecir el caudal natural en la cuenca hidrográfica Chambo (Ecuador) a partir de series de
entrada hidro-climáticas que abarcan un rango de 14 años. Se han considerados redes con
diferentes estructuras y se han entrenado de manera local, global y secuencial. Se ha medido el
desempeño de los modelos utilizando diferentes métricas y se ha comprobado que en la
mayoría de los casos, los modelos logran reconstruir de manera correcta los caudales en un
rango de 6 años. Con los resultados obtenidos se ha utilizado el software de gestión Modsim
que permite optimizar los usos de los recursos de la cuenca y se ha comprobado que los
resultados obtenidos por medio de los modelos de RN son los mismos que los obtenidos con un
modelo hidrológico tradicional. Finalmente se ha desarrollado un software de gestión que
ejecuta los modelos de forma distribuida y que considera los diferentes usos del agua.


\begin{itemize}
\item Item 1
\item Item 2
\end{itemize}